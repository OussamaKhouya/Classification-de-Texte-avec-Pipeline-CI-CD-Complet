% =============================================================================
% PRÉSENTATION MLOps - Classification de Texte
% Auteurs: Akram BENHAMMOU & Oussama KHOUYA
% Master 2 - DevOps & Machine Learning
% =============================================================================

\documentclass[aspectratio=169, 10pt]{beamer}
\setbeamersize{text margin left=10pt,text margin right=10pt}

% ============ PACKAGES ============
\usepackage[utf8]{inputenc}
\usepackage[T1]{fontenc}
\usepackage[french]{babel}
\usepackage{graphicx}
\usepackage{tikz}
\usepackage{fontawesome5}
\usepackage{xcolor}
\usepackage{listings}
\usepackage{booktabs}
\usepackage{hyperref}
\usepackage{multirow}

% ============ THÈME MODERNE PREMIUM ============
\usetheme{default}
\useinnertheme{rounded}

% ============ PALETTE DE COULEURS MODERNE ============
% Palette principale - Bleu professionnel avec accents vibrants
\definecolor{primaryDark}{RGB}{15, 23, 42}        % Slate 900 - Fond principal
\definecolor{primaryMid}{RGB}{30, 41, 59}         % Slate 800 - Fond secondaire
\definecolor{primaryLight}{RGB}{51, 65, 85}       % Slate 700 - Éléments subtils

% Couleurs d'accent modernes
\definecolor{accentCyan}{RGB}{6, 182, 212}        % Cyan 500 - Accent principal
\definecolor{accentTeal}{RGB}{20, 184, 166}       % Teal 500 - Accent secondaire
\definecolor{accentIndigo}{RGB}{99, 102, 241}     % Indigo 500 - Accent tertiaire

% Couleurs sémantiques pour DevOps/MLOps
\definecolor{devopsOrange}{RGB}{251, 146, 60}     % Orange 400 - DevOps
\definecolor{mlopsGreen}{RGB}{74, 222, 128}       % Green 400 - MLOps
\definecolor{dockerBlue}{RGB}{56, 189, 248}       % Sky 400 - Docker
\definecolor{githubPurple}{RGB}{167, 139, 250}    % Violet 400 - GitHub

% Couleurs de fond et texte
\definecolor{bgLight}{RGB}{248, 250, 252}         % Slate 50 - Fond clair
\definecolor{bgCard}{RGB}{241, 245, 249}          % Slate 100 - Cartes
\definecolor{textDark}{RGB}{15, 23, 42}           % Slate 900 - Texte principal
\definecolor{textMuted}{RGB}{100, 116, 139}       % Slate 500 - Texte secondaire

% ============ APPLICATION DU THÈME ============
% Cadre et titres
\setbeamercolor{frametitle}{fg=white,bg=primaryDark}
\setbeamercolor{framesubtitle}{fg=accentCyan}

% Structure générale
\setbeamercolor{structure}{fg=accentCyan}
\setbeamercolor{normal text}{fg=textDark,bg=bgLight}
\setbeamercolor{alerted text}{fg=devopsOrange}
\setbeamercolor{example text}{fg=mlopsGreen}

% Palette de navigation
\setbeamercolor{palette primary}{bg=primaryDark,fg=white}
\setbeamercolor{palette secondary}{bg=primaryMid,fg=white}
\setbeamercolor{palette tertiary}{bg=accentCyan,fg=primaryDark}
\setbeamercolor{palette quaternary}{bg=accentIndigo,fg=white}

% Blocs stylisés
\setbeamercolor{block title}{bg=accentCyan,fg=primaryDark}
\setbeamercolor{block body}{bg=bgCard,fg=textDark}
\setbeamercolor{block title alerted}{bg=devopsOrange,fg=primaryDark}
\setbeamercolor{block body alerted}{bg=bgCard,fg=textDark}
\setbeamercolor{block title example}{bg=mlopsGreen,fg=primaryDark}
\setbeamercolor{block body example}{bg=bgCard,fg=textDark}

% Items et puces
\setbeamercolor{item}{fg=accentCyan}
\setbeamercolor{subitem}{fg=accentTeal}
\setbeamercolor{itemize item}{fg=accentCyan}
\setbeamercolor{itemize subitem}{fg=accentTeal}
\setbeamercolor{enumerate item}{fg=accentCyan}

% Page de titre
\setbeamercolor{title}{fg=white,bg=primaryDark}
\setbeamercolor{subtitle}{fg=accentCyan}
\setbeamercolor{author}{fg=black}
\setbeamercolor{institute}{fg=accentTeal}
\setbeamercolor{date}{fg=textMuted}

% Footer élégant
\setbeamercolor{footline}{fg=textMuted,bg=bgLight}

% ============ TYPOGRAPHIE MODERNE ============
\setbeamerfont{title}{size=\LARGE,series=\bfseries}
\setbeamerfont{subtitle}{size=\normalsize,series=\mdseries}
\setbeamerfont{frametitle}{size=\large,series=\bfseries}
\setbeamerfont{framesubtitle}{size=\small,series=\mdseries}
\setbeamerfont{block title}{size=\normalsize,series=\bfseries}
\setbeamerfont{author}{size=\normalsize,series=\mdseries}
\setbeamerfont{institute}{size=\small}
\setbeamerfont{date}{size=\small}

% ============ TEMPLATES PERSONNALISÉS ============
% Supprimer les éléments de navigation par défaut
\setbeamertemplate{navigation symbols}{}

% Footer minimaliste avec numéro de page
\setbeamertemplate{footline}{
    \hfill\usebeamercolor[fg]{footline}\insertframenumber\,/\,\inserttotalframenumber\hspace*{4pt}\vskip4pt
}

% Puces modernes
\setbeamertemplate{itemize item}{\textcolor{accentCyan}{\faChevronRight}}
\setbeamertemplate{itemize subitem}{\textcolor{accentTeal}{\faAngleRight}}
\setbeamertemplate{itemize subsubitem}{\textcolor{textMuted}{\textbullet}}

% Numérotation colorée
\setbeamertemplate{enumerate item}{\textcolor{accentCyan}{\insertenumlabel.}}

% Blocs arrondis avec ombre
\setbeamertemplate{blocks}[rounded][shadow=true]

% ============ STYLE DES LISTINGS (CODE) ============
\lstset{
    basicstyle=\ttfamily\scriptsize,
    backgroundcolor=\color{bgCard},
    frame=l,
    framesep=10pt,
    framerule=3pt,
    rulecolor=\color{accentCyan},
    keywordstyle=\color{accentIndigo}\bfseries,
    commentstyle=\color{textMuted}\itshape,
    stringstyle=\color{mlopsGreen},
    breaklines=true,
    showstringspaces=false,
    numbers=none,
    xleftmargin=15pt
}

% ============ INFORMATIONS ============
\title[MLOps Pipeline]{{\textbf{Pipeline MLOps Complet}\\Classification Automatique de Texte}}
\subtitle{GitHub Actions • Docker • MLflow • CML • FastAPI • Angular}

\author[BENHAMMOU \& KHOUYA]{%
\textbf{Réalisé par :}\\
Akram BENHAMMOU \&
Oussama KHOUYA
}

\institute[M2 DevOps]{MLOPS \& DEVOPS}
\date{15/12/2025}

\begin{document}


% -----------------------------------------------------------------------------
% SLIDE 1: PAGE DE TITRE
% -----------------------------------------------------------------------------
\begin{frame}[plain]
    \titlepage
\end{frame}

% -----------------------------------------------------------------------------
% SLIDE 2: SOMMAIRE
% -----------------------------------------------------------------------------
\begin{frame}{Sommaire}
    \tableofcontents
\end{frame}

% =============================================================================
\section{Introduction \& Contexte}
% =============================================================================

% -----------------------------------------------------------------------------
% SLIDE 3: CONTEXTE
% -----------------------------------------------------------------------------
\begin{frame}{Contexte du Projet}
    \begin{columns}
        \column{0.55\textwidth}
        \begin{block}{Problématique}
            Comment automatiser le cycle de vie complet d'un modèle de Machine Learning, de l'entraînement au déploiement en production ?
        \end{block}
        
        \vspace{0.3cm}
        
        \textbf{Cas d'usage :} Classification automatique d'articles de journaux
        
        \vspace{0.3cm}
        
        \begin{alertblock}{Objectifs du projet}
            \begin{itemize}
                \item[\faCheckCircle] Pipeline \textbf{CI/CD} automatisé
                \item[\faCheckCircle] Tracking d'expériences \textbf{MLflow}
                \item[\faCheckCircle] Conteneurisation \textbf{Docker}
                \item[\faCheckCircle] Déploiement \textbf{Staging → Production}
            \end{itemize}
        \end{alertblock}
        
        \column{0.45\textwidth}
        \centering
        \begin{tikzpicture}[node distance=1.5cm]
            \node[draw, fill=devopsOrange!30, rounded corners, minimum width=3.5cm, minimum height=0.9cm, font=\small] (push) {\faCode\ Code Push};
            \node[draw, fill=githubPurple!30, rounded corners, minimum width=3.5cm, minimum height=0.9cm, font=\small, below of=push] (actions) {\faGithub\ GitHub Actions};
            \node[draw, fill=dockerBlue!30, rounded corners, minimum width=3.5cm, minimum height=0.9cm, font=\small, below of=actions] (docker) {\faDocker\ Docker Image};
            \node[draw, fill=mlopsGreen!30, rounded corners, minimum width=3.5cm, minimum height=0.9cm, font=\small, below of=docker] (prod) {\faRocket\ Production};
            
            \draw[->, thick, accentCyan] (push) -- (actions);
            \draw[->, thick, accentCyan] (actions) -- (docker);
            \draw[->, thick, accentCyan] (docker) -- (prod);
        \end{tikzpicture}
    \end{columns}
\end{frame}

% -----------------------------------------------------------------------------
% SLIDE 4: LES 7 CATÉGORIES
% -----------------------------------------------------------------------------
\begin{frame}{Les 7 Catégories de Classification}
    \centering
    \begin{tabular}{ccc}
        \colorbox{accentCyan!30}{\makebox[3cm]{\faLaptop\ Informatique}} &
        \colorbox{green!30}{\makebox[3cm]{\faFutbol\ Sport}} &
        \colorbox{yellow!30}{\makebox[3cm]{\faFlask\ Science}} \\[0.5cm]
        \colorbox{red!30}{\makebox[3cm]{\faLandmark\ Politique}} &
        \colorbox{purple!30}{\makebox[3cm]{\faChurch\ Religion}} &
        \colorbox{orange!30}{\makebox[3cm]{\faCar\ Automobile}} \\[0.5cm]
        \multicolumn{3}{c}{\colorbox{cyan!30}{\makebox[3cm]{\faShoppingCart\ Commerce}}}
    \end{tabular}
    
    \vspace{0.8cm}
    
    \begin{block}{Dataset : 20 Newsgroups}
        \begin{itemize}
            \item \textbf{18,846 articles} répartis en 20 classes originales
            \item Regroupement en \textbf{7 catégories} pour simplification
            \item Split : 80\% entraînement / 20\% test
        \end{itemize}
    \end{block}
\end{frame}

% =============================================================================
\section{Architecture Technique}
% =============================================================================

% -----------------------------------------------------------------------------
% SLIDE 5: ARCHITECTURE GLOBALE
% -----------------------------------------------------------------------------
\begin{frame}{Architecture Globale du Système}
    \centering
    \begin{tikzpicture}[
        node distance=1.2cm,
        box/.style={draw, rounded corners, fill=#1, minimum width=2.2cm, minimum height=1cm, align=center, font=\small},
        arrow/.style={->, thick, accentCyan}
    ]
        % Layer 1: Data & ML
        \node[box=yellow!20] (data) at (0,0) {\faDatabase\\ Données};
        \node[box=orange!20] (preprocess) at (3.2,0) {\faCogs\\ Prétraitement};
        \node[box=mlopsGreen!20] (train) at (6.4,0) {\faBrain\\ Entraînement};
        
        % Layer 2: MLOps
        \node[box=purple!20] (mlflow) at (6.4,-1.8) {\faChartLine\\ MLflow};
        
        % Layer 3: Deployment
        \node[box=accentCyan!20] (api) at (9.6,0) {\faServer\\ FastAPI};
        \node[box=dockerBlue!20] (docker) at (9.6,-1.8) {\faDocker\\ Docker};
        
        % Layer 4: Frontend
        \node[box=red!20] (frontend) at (12.8,0) {\faDesktop\\ Angular};
        
        % Arrows
        \draw[arrow] (data) -- (preprocess);
        \draw[arrow] (preprocess) -- (train);
        \draw[arrow] (train) -- (api);
        \draw[arrow] (train) -- (mlflow);
        \draw[arrow] (api) -- (docker);
        \draw[arrow] (api) -- (frontend);
    \end{tikzpicture}
    
    \vspace{0.5cm}
    
    \begin{columns}
        \column{0.25\textwidth}
        \centering
        \textbf{Data Pipeline}\\
        \footnotesize NLTK, pandas\\TF-IDF
        
        \column{0.25\textwidth}
        \centering
        \textbf{ML Training}\\
        \footnotesize scikit-learn\\Random Forest
        
        \column{0.25\textwidth}
        \centering
        \textbf{MLOps}\\
        \footnotesize MLflow, CML\\Tracking
        
        \column{0.25\textwidth}
        \centering
        \textbf{Deployment}\\
        \footnotesize Docker, GHCR\\FastAPI
    \end{columns}
\end{frame}

% =============================================================================
\section{Outils DevOps}
% =============================================================================

% -----------------------------------------------------------------------------
% SLIDE 6: GITHUB ACTIONS - VUE D'ENSEMBLE
% -----------------------------------------------------------------------------
\begin{frame}{GitHub Actions - Pipeline CI/CD}
    \centering
    \textbf{\faGithub\ Automatisation complète avec 3 workflows}
    
    \vspace{0.5cm}
    
    \begin{tikzpicture}[
        node distance=0.6cm,
        workflow/.style={draw, rounded corners, fill=#1, minimum width=4cm, minimum height=1.2cm, align=center, font=\small},
        arrow/.style={->, thick, primaryBlue}
    ]
        \node[workflow=yellow!30] (push) at (0,2) {\faCodeBranch\ \texttt{git push master}};
        
        \node[workflow=cyan!30] (cml) at (-4.5,0) {\texttt{cml.yaml}\\ \faChartBar\ Rapport Métriques};
        \node[workflow=dockerBlue!30] (docker) at (0,0) {\texttt{docker.yaml}\\ \faDocker\ Build \& Push};
        \node[workflow=mlopsGreen!30] (deploy) at (4.5,0) {\texttt{deploy.yaml}\\ \faRocket\ Staging → Prod};
        
        \draw[arrow] (push) -- (cml);
        \draw[arrow] (push) -- (docker);
        \draw[arrow] (docker) -- node[above, font=\tiny] {trigger} (deploy);
    \end{tikzpicture}
    
    \vspace{0.3cm}
    
    \begin{columns}
        \column{0.33\textwidth}
        \begin{block}{\footnotesize 1. CML Report}
            \footnotesize
            \begin{itemize}
                \item Entraînement modèle
                \item Génération métriques
                \item Commentaire PR auto
            \end{itemize}
        \end{block}
        
        \column{0.33\textwidth}
        \begin{block}{\footnotesize 2. Docker Build}
            \footnotesize
            \begin{itemize}
                \item Tests pytest
                \item Build image
                \item Push vers GHCR
            \end{itemize}
        \end{block}
        
        \column{0.33\textwidth}
        \begin{block}{\footnotesize 3. Deploy Pipeline}
            \footnotesize
            \begin{itemize}
                \item Deploy Staging
                \item Tests d'intégration
                \item Deploy Production
            \end{itemize}
        \end{block}
    \end{columns}
\end{frame}

% -----------------------------------------------------------------------------
% SLIDE 7: WORKFLOW CML DÉTAILLÉ
% -----------------------------------------------------------------------------
\begin{frame}[fragile]{Workflow CML - Continuous Machine Learning}
    \begin{columns}
        \column{0.5\textwidth}
        \textbf{\faFileCode\ cml.yaml - Étapes clés}
        
        \vspace{0.3cm}
        
        \begin{lstlisting}[language=yaml]
name: CML Report
on:
  push:
    branches: [ master ]
    
jobs:
  train-and-report:
    steps:
      - uses: iterative/setup-cml@v2
      
      - name: Preprocess Data
        run: python src/preprocess.py
        
      - name: Train Model
        run: python src/train.py
        
      - name: Write CML Report
        run: |
          cml publish reports/...
          cml comment create report.md
        \end{lstlisting}
        
        \column{0.5\textwidth}
        \textbf{\faClipboardCheck\ Fonctionnalités CML}
        
        \vspace{0.3cm}
        
        \begin{block}{Rapport Automatique}
            \begin{itemize}
                \item[\faCog] Setup Python 3.9
                \item[\faDatabase] Prétraitement données
                \item[\faBrain] Entraînement modèle
                \item[\faChartBar] Génération métriques JSON
                \item[\faImage] Matrice de confusion
                \item[\faComment] Commentaire GitHub auto
            \end{itemize}
        \end{block}
        
        \vspace{0.3cm}
        
        \begin{alertblock}{\footnotesize Outil: iterative/cml}
            \footnotesize Permet de publier des rapports ML directement dans les Pull Requests GitHub
        \end{alertblock}
    \end{columns}
\end{frame}

% -----------------------------------------------------------------------------
% SLIDE 8: WORKFLOW DOCKER DÉTAILLÉ
% -----------------------------------------------------------------------------
\begin{frame}[fragile]{Workflow Docker - Build \& Push}
    \begin{columns}
        \column{0.48\textwidth}
        \textbf{\faDocker\ docker.yaml - Pipeline}
        
        \vspace{0.2cm}
        
        \begin{lstlisting}[language=yaml]
name: Docker Build & Push
env:
  REGISTRY: ghcr.io
  IMAGE_NAME: ${{ github.repository }}

jobs:
  build-and-push:
    steps:
      - name: Run tests
        run: pytest tests/ -v
        
      - name: Log in to GHCR
        uses: docker/login-action@v3
        
      - name: Build and push
        uses: docker/build-push-action@v5
        with:
          push: true
          tags: |
            type=sha
            type=raw,value=latest
        \end{lstlisting}
        
        \column{0.52\textwidth}
        \textbf{\faLayerGroup\ Dockerfile Optimisé}
        
        \vspace{0.2cm}
        
        \begin{lstlisting}
FROM python:3.9-slim

WORKDIR /app

# Cache des dependances
COPY requirements.txt .
RUN pip install --no-cache-dir \
    -r requirements.txt

# Ressources NLTK
RUN python -c "import nltk; \
    nltk.download('stopwords'); \
    nltk.download('wordnet')"

COPY src/ ./src/
COPY models/ ./models/

EXPOSE 8000

CMD ["uvicorn", "src.app:app", \
     "--host", "0.0.0.0", "--port", "8000"]
        \end{lstlisting}
    \end{columns}
\end{frame}

% -----------------------------------------------------------------------------
% SLIDE 9: WORKFLOW DEPLOY DÉTAILLÉ
% -----------------------------------------------------------------------------
\begin{frame}{Workflow Deploy - Staging → Production}
    \centering
    \begin{tikzpicture}[
        node distance=1cm,
        stage/.style={draw, rounded corners, fill=#1, minimum width=3.5cm, minimum height=1.5cm, align=center, font=\small},
        arrow/.style={->, thick, primaryBlue},
        decision/.style={diamond, draw, fill=yellow!30, minimum width=1.5cm, minimum height=1cm, align=center, font=\tiny}
    ]
        % Trigger
        \node[stage=githubPurple!30] (trigger) at (0,0) {\faCodeBranch\\ Docker Build Success};
        
        % Staging
        \node[stage=orange!30] (staging) at (4.5,0) {\faServer\ Deploy Staging\\ \footnotesize docker run -p 8000:8000};
        
        % Tests
        \node[stage=cyan!30] (tests) at (9,0) {\faVial\ Tests Intégration\\ \footnotesize curl /health, /predict};
        
        % Decision
        \node[decision] (check) at (9,-2) {Accuracy\\>50\%?};
        
        % Production
        \node[stage=mlopsGreen!30] (prod) at (4.5,-2) {\faRocket\ Deploy Production\\ \footnotesize Tag release};
        
        % Rollback
        \node[stage=red!30] (rollback) at (12,-2) {\faUndo\ Rollback\\ \footnotesize Créer Issue};
        
        % Arrows
        \draw[arrow] (trigger) -- (staging);
        \draw[arrow] (staging) -- (tests);
        \draw[arrow] (tests) -- (check);
        \draw[arrow] (check) -- node[left, font=\tiny] {Oui} (prod);
        \draw[arrow] (check) -- node[above, font=\tiny] {Non} (rollback);
    \end{tikzpicture}
    
    \vspace{0.5cm}
    
    \begin{columns}
        \column{0.5\textwidth}
        \begin{block}{\footnotesize Tests d'intégration Staging}
            \footnotesize
            \begin{itemize}
                \item \texttt{GET /health} → model\_loaded: true
                \item \texttt{POST /predict} → Test prédiction
                \item Seuil accuracy minimum : 50\%
            \end{itemize}
        \end{block}
        
        \column{0.5\textwidth}
        \begin{alertblock}{\footnotesize Rollback Automatique}
            \footnotesize
            \begin{itemize}
                \item Restauration version précédente
                \item Création Issue GitHub automatique
                \item Notification équipe
            \end{itemize}
        \end{alertblock}
    \end{columns}
\end{frame}

% -----------------------------------------------------------------------------
% SLIDE 10: TESTS AUTOMATISÉS
% -----------------------------------------------------------------------------
\begin{frame}[fragile]{Tests Automatisés - pytest}
    \textbf{\faVial\ Suite de tests complète exécutée dans le CI/CD}
    
    \vspace{0.3cm}
    
    \begin{columns}
        \column{0.33\textwidth}
        \begin{block}{\footnotesize test\_preprocess.py}
            \footnotesize Tests du prétraitement
            \begin{itemize}
                \item clean\_text()
                \item process\_text()
                \item Gestion stopwords
                \item Lemmatisation
            \end{itemize}
        \end{block}
        
        \column{0.33\textwidth}
        \begin{block}{\footnotesize test\_train.py}
            \footnotesize Tests d'entraînement
            \begin{itemize}
                \item Chargement données
                \item TF-IDF vectorizer
                \item RandomForest fit
                \item Sauvegarde modèle
            \end{itemize}
        \end{block}
        
        \column{0.33\textwidth}
        \begin{block}{\footnotesize test\_api.py}
            \footnotesize Tests API FastAPI
            \begin{itemize}
                \item GET /health
                \item POST /predict
                \item POST /upload
                \item Codes erreur
            \end{itemize}
        \end{block}
    \end{columns}
    
    \vspace{0.3cm}
    
    \begin{lstlisting}[language=bash]
# Execution dans le workflow docker.yaml
- name: Run tests
  run: pytest tests/ -v --tb=short
    \end{lstlisting}
\end{frame}

% =============================================================================
\section{Outils MLOps}
% =============================================================================

% -----------------------------------------------------------------------------
% SLIDE 11: MLFLOW TRACKING
% -----------------------------------------------------------------------------
\begin{frame}[fragile]{MLflow - Tracking d'Expériences}
    \begin{columns}
        \column{0.5\textwidth}
        \textbf{\faChartLine\ Intégration MLflow}
        
        \begin{lstlisting}[language=python,basicstyle=\ttfamily\tiny]
import mlflow

mlflow.set_tracking_uri("file:./mlruns")
mlflow.set_experiment("Text_Classification")

with mlflow.start_run():
    # Hyperparametres
    mlflow.log_param("max_features", 5000)
    mlflow.log_param("n_estimators", 100)
    
    # Metriques
    mlflow.log_metric("accuracy", accuracy)
    mlflow.log_metric("f1_score", f1)
    
    # Artefacts
    mlflow.sklearn.log_model(model, "model")
        \end{lstlisting}
        
        \column{0.5\textwidth}
        \textbf{\faDatabase\ Éléments trackés}
        
        \begin{block}{\scriptsize Paramètres}
            \scriptsize
            \begin{itemize}
                \item \texttt{max\_features}: 5000
                \item \texttt{n\_estimators}: 100
            \end{itemize}
        \end{block}
        
        \begin{block}{\scriptsize Métriques}
            \scriptsize
            \begin{itemize}
                \item Accuracy, Precision
                \item Recall, F1-Score
            \end{itemize}
        \end{block}
        
        \begin{block}{\scriptsize Artefacts}
            \scriptsize
            \begin{itemize}
                \item Modèle 
                \item TF-IDF Vectorizer
                \item Matrice de confusion
            \end{itemize}
        \end{block}
    \end{columns}
\end{frame}

% -----------------------------------------------------------------------------
% SLIDE 12: CML - REPORTING
% -----------------------------------------------------------------------------
\begin{frame}{CML - Continuous Machine Learning}
    \begin{columns}
        \column{0.5\textwidth}
        \textbf{\faComment\ Rapport GitHub Automatique}
        
        \vspace{0.3cm}
        
        \begin{block}{\scriptsize Contenu du rapport}
            \scriptsize
            \begin{enumerate}
                \item Metriques JSON (accuracy, f1)
                \item Rapport par classe (txt)
                \item Matrice confusion (image)
            \end{enumerate}
        \end{block}
        
        \begin{alertblock}{\scriptsize Avantages CML}
            \scriptsize
            \begin{itemize}
                \item Visibilite dans PR
                \item Historique performances
            \end{itemize}
        \end{alertblock}
        
        \column{0.5\textwidth}
        \centering
        \textbf{\small Exemple de rapport}
        
        \vspace{0.2cm}
        
        \includegraphics[width=5cm]{screenshots/rapport-cml-genere.png}
    \end{columns}
\end{frame}

% =============================================================================
\section{Backend \& Frontend}
% =============================================================================

% -----------------------------------------------------------------------------
% SLIDE 13: FASTAPI BACKEND
% -----------------------------------------------------------------------------
\begin{frame}[fragile]{FastAPI - Backend API REST}
    \begin{columns}
        \column{0.48\textwidth}
        \textbf{\faServer\ Endpoints de l'API}
        
        \vspace{0.2cm}
        
        \begin{tabular}{ll}
            \toprule
            \textbf{Méthode} & \textbf{Endpoint} \\
            \midrule
            GET & \texttt{/health} \\
            POST & \texttt{/predict} \\
            POST & \texttt{/upload} \\
            GET & \texttt{/docs} \\
            \bottomrule
        \end{tabular}
        
        \vspace{0.3cm}
        
        \begin{block}{\footnotesize Fonctionnalités}
            \begin{itemize}
                \item[\faFileAlt] Support PDF, DOCX, TXT
                \item[\faPercent] Scores de confiance
                \item[\faHeartbeat] Health check
                \item[\faBook] Swagger auto (OpenAPI)
                \item[\faGlobe] CORS configuré
            \end{itemize}
        \end{block}
        
        \column{0.52\textwidth}
        \textbf{\faCode\ Exemple de réponse /predict}
        
        \vspace{0.2cm}
        
        \begin{lstlisting}[language=json]
{
  "text": "Computer graphics and...",
  "prediction_class_id": 4,
  "category_name": "Informatique",
  "confidence_scores": [
    {"name": "Informatique", "value": 0.72},
    {"name": "Science", "value": 0.15},
    {"name": "Commerce", "value": 0.08},
    ...
  ],
  "status": "success"
}
        \end{lstlisting}
        
        \vspace{0.2cm}
        
        \begin{alertblock}{\footnotesize Lifespan Pattern}
            \footnotesize Chargement du modèle au démarrage via \texttt{@asynccontextmanager}
        \end{alertblock}
    \end{columns}
\end{frame}

% -----------------------------------------------------------------------------
% SLIDE 14: ANGULAR FRONTEND
% -----------------------------------------------------------------------------
\begin{frame}{Angular 21 - Interface Utilisateur}
    \centering
    \textbf{Application Web Moderne}
    
    \vspace{0.1cm}
    
    \includegraphics[width=6cm,height=3.5cm,keepaspectratio]{screenshots/ui-demo.png}
    
    \vspace{0.1cm}
    
    \begin{columns}
        \column{0.5\textwidth}
        \begin{block}{\footnotesize Fonctionnalités}
            \begin{itemize}
                \item Saisie de texte libre
                \item Import fichiers (PDF, DOCX, txt,md...)
                \item Barres de classes dynamiques
            \end{itemize}
        \end{block}
        
        \column{0.5\textwidth}
        \begin{block}{\footnotesize Technologies}
            \begin{itemize}
                \item Angular 21
                \item Design moderne
                \item HttpClient pour API
            \end{itemize}
        \end{block}
    \end{columns}
\end{frame}

% =============================================================================
\section{Démonstration}
% =============================================================================

% -----------------------------------------------------------------------------
% SLIDE 15: DÉMO DOCKER
% -----------------------------------------------------------------------------
\begin{frame}[fragile]{Démonstration - Docker en Local}
    \textbf{\faDocker\ Lancer l'application conteneurisée}
    
    \vspace{0.1cm}
    
    \begin{columns}
        \column{0.5\textwidth}
        \begin{block}{\scriptsize 1. Pull \& Run}
            \begin{lstlisting}[language=bash,basicstyle=\ttfamily\tiny]
docker pull ghcr.io/akrambenhammou-e/\
  classification-texte-pipeline-ci-cd:latest
docker run -d -p 8000:8000 --name app \
  ghcr.io/.../...:latest
            \end{lstlisting}
        \end{block}
        
        \column{0.5\textwidth}
        \begin{block}{\scriptsize 2. Tester l'API}
            \begin{lstlisting}[language=bash,basicstyle=\ttfamily\tiny]
# Health check
curl http://localhost:8000/health

# Prediction
curl -X POST localhost:8000/predict \
  -H "Content-Type: application/json" \
  -d '{"text": "GPU NVIDIA gaming"}'
            \end{lstlisting}
        \end{block}
    \end{columns}
\end{frame}

% -----------------------------------------------------------------------------
% SLIDE 16: DÉMO LIVE
% -----------------------------------------------------------------------------
\begin{frame}{Démonstration Live}
    \centering
    
    \begin{block}{Étapes de la démonstration}
        \begin{enumerate}
            \item Lancer l'API : \texttt{uvicorn src.app:app --reload}
            \item Lancer le frontend : \texttt{cd frontend \&\& ng serve}
            \item Ouvrir \url{http://localhost:4200}
            \item Tester avec un article en anglais
            \item Observer les probabilités par catégorie
        \end{enumerate}
    \end{block}
    
    \vspace{0.5cm}
    
    \begin{columns}
        \column{0.33\textwidth}
        \centering
        \textbf{API Backend}\\
        \url{http://localhost:8000}\\
        \footnotesize Swagger: \texttt{/docs}
        
        \column{0.33\textwidth}
        \centering
        \textbf{Frontend Angular}\\
        \url{http://localhost:4200}\\
        \footnotesize Interface utilisateur
        
        \column{0.33\textwidth}
        \centering
        \textbf{MLflow UI}\\
        \texttt{mlflow ui}\\
        \footnotesize Port 5000
    \end{columns}
\end{frame}

% =============================================================================
\section{Résultats \& Métriques}
% =============================================================================

% -----------------------------------------------------------------------------
% SLIDE 17: MÉTRIQUES DU MODÈLE
% -----------------------------------------------------------------------------
\begin{frame}{Performances du Modèle}
    \begin{columns}
        \column{0.5\textwidth}
        \centering
        \textbf{Métriques d'évaluation}
        
        \vspace{0.5cm}
        
        \begin{tabular}{lc}
            \toprule
            \textbf{Métrique} & \textbf{Score} \\
            \midrule
            Accuracy & 64\% \\
            Precision (weighted) & 63\% \\
            Recall (weighted) & 64\% \\
            F1-Score (weighted) & 63\% \\
            \bottomrule
        \end{tabular}
        
        \vspace{0.5cm}
        
        \begin{block}{\footnotesize Configuration Modèle}
            \begin{itemize}
                \footnotesize
                \item RandomForest (100 arbres)
                \item TF-IDF (5000 features)
                \item Train/Test: 80/20
            \end{itemize}
        \end{block}
        
        \column{0.5\textwidth}
        \centering
        \textbf{Matrice de Confusion}
        
        % Placeholder pour image
        \includegraphics[width=5.5cm]{screenshots/confusion_matrix.png}
    \end{columns}
\end{frame}

% =============================================================================
\section{Conclusion}
% =============================================================================

% -----------------------------------------------------------------------------
% SLIDE 18: STACK TECHNOLOGIQUE
% -----------------------------------------------------------------------------
\begin{frame}{Stack Technologique Complète}
    \centering
    \begin{tabular}{lll}
        \toprule
        \textbf{Catégorie} & \textbf{Outils} & \textbf{Rôle} \\
        \midrule
        \multirow{2}{*}{\textbf{DevOps}} & GitHub Actions & CI/CD automatisé \\
        & Docker, GHCR & Conteneurisation \\
        \midrule
        \multirow{2}{*}{\textbf{MLOps}} & MLflow & Tracking expériences \\
        & CML & Rapports ML dans PR \\
        \midrule
        \multirow{2}{*}{\textbf{ML}} & scikit-learn & Modèle RandomForest \\
        & NLTK, pandas & Prétraitement NLP \\
        \midrule
        \multirow{2}{*}{\textbf{Backend}} & FastAPI & API REST \\
        & Uvicorn, Pydantic & Serveur ASGI \\
        \midrule
        \textbf{Frontend} & Angular 21 & Interface utilisateur \\
        \midrule
        \textbf{Tests} & pytest, httpx & Tests automatisés \\
        \bottomrule
    \end{tabular}
\end{frame}

% -----------------------------------------------------------------------------
% SLIDE 21: CONCLUSION
% -----------------------------------------------------------------------------
\begin{frame}{Conclusion \& Perspectives}
    \begin{columns}
        \column{0.5\textwidth}
        \begin{block}{\faCheckCircle\ Réalisations}
            \begin{itemize}
                \item Pipeline \textbf{CI/CD complet}
                    \begin{itemize}
                        \footnotesize
                        \item 3 workflows GitHub Actions
                        \item Tests automatisés (pytest)
                        \item Rollback automatique
                    \end{itemize}
                \item \textbf{MLOps} intégré
                    \begin{itemize}
                        \footnotesize
                        \item MLflow tracking
                        \item CML reporting
                    \end{itemize}
                \item \textbf{Déploiement} Docker
                    \begin{itemize}
                        \footnotesize
                        \item Image optimisée
                        \item GitHub Container Registry
                    \end{itemize}
            \end{itemize}
        \end{block}
        
        \column{0.5\textwidth}
        \begin{alertblock}{\faRocket\ Améliorations possibles}
            \begin{itemize}
                \item \textbf{Modèle}
                    \begin{itemize}
                        \footnotesize
                        \item Deep Learning (BERT)
                        \item Support multilingue
                    \end{itemize}
                \item \textbf{Infrastructure}
                    \begin{itemize}
                        \footnotesize
                        \item Kubernetes (scaling)
                        \item Prometheus (monitoring)
                    \end{itemize}
                \item \textbf{MLOps avancé}
                    \begin{itemize}
                        \footnotesize
                        \item A/B Testing
                        \item Feature Store
                        \item Model Registry
                    \end{itemize}
            \end{itemize}
        \end{alertblock}
    \end{columns}
\end{frame}

% -----------------------------------------------------------------------------
% SLIDE 20: QUESTIONS
% -----------------------------------------------------------------------------
\begin{frame}[plain]
    \centering
    
    \vspace{2cm}
    
    {\Huge \textbf{Merci de votre attention !}}
    
    \vspace{1cm}
    
    {\Large \faQuestionCircle\ Questions ?}
    
    \vspace{1.5cm}
    
    \begin{tabular}{cc}
        \faGithub\ & \url{github.com/AkramBENHAMMOU-e/Classification-Texte-Pipeline-CI-CD} \\
    \end{tabular}
    
    \vspace{1cm}
    
    \textit{Akram BENHAMMOU \& Oussama KHOUYA}\\
    \small Master 2 - AI \& Systemes distrubués
\end{frame}

\end{document}
\documentclass[aspectratio=169]{beamer}

\usepackage[utf8]{inputenc}
\usepackage[T1]{fontenc}
\usepackage[french]{babel}
\usepackage{graphicx}

\usetheme{Madrid}

\title{Classification de texte avec pipeline CI/CD complet}
\subtitle{Projet DevOps \& MLOps}
\author{Akram Benhammou \and Oussama Khouya}
\institute{ENSET / Université Hassan II}
\date{\today}

\begin{document}

\begin{frame}
  \titlepage
\end{frame}

\begin{frame}{Plan}
  \tableofcontents
\end{frame}

\section{Contexte et objectifs}

\begin{frame}{Contexte du projet}
  \begin{itemize}
    \item Problème de classification de texte sur le jeu de données \emph{20 Newsgroups}.
    \item Mise en place d'une chaîne MLOps de bout en bout.
    \item Intégration des bonnes pratiques DevOps : CI/CD, tests, containerisation.
  \end{itemize}
\end{frame}

\begin{frame}{Objectifs}
  \begin{itemize}
    \item Automatiser le cycle de vie du modèle (données $\rightarrow$ modèle $\rightarrow$ déploiement).
    \item Garantir la reproductibilité des expériences.
    \item Faciliter le déploiement et la maintenance du service de prédiction.
  \end{itemize}
\end{frame}

\section{Dataset et prétraitement}

\begin{frame}{Dataset 20 Newsgroups}
  \begin{itemize}
    \item 20 classes de newsgroups (informatique, sport, politique, religion, \dots).
    \item Chargement via \texttt{sklearn.datasets.fetch\_20newsgroups}.
    \item Séparation en train / validation / test (70\% / 15\% / 15\%).
  \end{itemize}
\end{frame}

\begin{frame}{Pipeline de données}
  \begin{itemize}
    \item Nettoyage simple du texte brut (minuscules, ponctuation, chiffres).
    \item Tokenisation, suppression des stopwords anglais, lemmatisation (NLTK).
    \item Vectorisation TF--IDF (5000 features max).
  \end{itemize}
  \begin{center}
    \includegraphics[width=0.9\textwidth]{snapshots/pipeline_horizontal.png}
  \end{center}
\end{frame}

\section{Modèle et suivi expérimental}

\begin{frame}{Modèle de classification}
  \begin{itemize}
    \item \texttt{RandomForestClassifier} sur les vecteurs TF--IDF.
    \item Entraînement sur le jeu d'apprentissage, évaluation sur le test.
    \item Métriques : accuracy, précision, rappel, F1 pondérée, matrice de confusion.
  \end{itemize}
  \begin{center}
    \includegraphics[width=0.9\textwidth]{snapshots/train_eval_horizontal.png}
  \end{center}
\end{frame}

\begin{frame}{Suivi avec MLflow}
  \begin{itemize}
    \item Backend local \texttt{mlruns/}.
    \item Logging des hyperparamètres, métriques et artefacts (modèle, vectoriseur).
    \item Possibilité de comparer plusieurs expériences.
  \end{itemize}
\end{frame}

\section{API, Docker et CI/CD}

\begin{frame}{Service d'inférence FastAPI}
  \begin{itemize}
    \item Endpoint \texttt{/health} pour la supervision.
    \item Endpoint \texttt{/predict} pour la prédiction sur un texte.
    \item Chargement des artefacts au démarrage de l'application.
  \end{itemize}
  \begin{center}
    \includegraphics[width=0.5\textwidth]{snapshots/fastapi-predict.png}
  \end{center}
\end{frame}

\begin{frame}{Containerisation Docker}
  \begin{itemize}
    \item Image basée sur \texttt{python:3.9-slim}.
    \item Installation des dépendances et des ressources NLTK.
    \item Exposition du service via Uvicorn sur le port 8000.
  \end{itemize}
  \begin{center}
    \includegraphics[width=1\textwidth]{snapshots/docker_build_horizontal.png}
  \end{center}
\end{frame}

\begin{frame}{CI/CD GitHub Actions}
  \begin{itemize}
    \item Workflow de build et tests : prétraitement, entraînement, tests, build/push Docker.
    \item Workflow CML : génération automatique d'un rapport de performance.
    \item Workflow de déploiement : staging $\rightarrow$ production avec rollback.
  \end{itemize}
  \begin{center}
    \includegraphics[width=0.9\textwidth]{snapshots/cml_workflow_horizontal.png}
    \hspace{0.5cm}
    \includegraphics[width=0.9\textwidth]{snapshots/deploy_pipeline_horizontal.png}
  \end{center}
\end{frame}

\section{Limites et perspectives}

\begin{frame}{Limites}
  \begin{itemize}
    \item Modèle classique (Random Forest + TF--IDF) sans transformers.
    \item MLflow en mode local, sans model registry centralisé.
    \item Monitoring de la dérive de données et des performances en production non implémenté.
  \end{itemize}
\end{frame}

\begin{frame}{Pistes d'amélioration}
  \begin{itemize}
    \item Expérimenter des modèles basés sur des embeddings ou des transformers.
    \item Mettre en place un registry de modèles et un tracking centralisé.
    \item Ajouter du monitoring, des alertes et des tests de robustesse avancés.
  \end{itemize}
\end{frame}

\section{Conclusion}

\begin{frame}{Conclusion}
  \begin{itemize}
    \item Mise en œuvre d'un pipeline MLOps complet autour d'un cas réel de NLP.
    \item Automatisation du cycle de vie du modèle avec GitHub Actions, Docker et MLflow.
    \item Base solide pour des évolutions vers des architectures et modèles plus avancés.
  \end{itemize}
\end{frame}

\begin{frame}{Questions ?}
  \centering
  Merci pour votre attention !
\end{frame}

\end{document}
